\documentclass{article}
\usepackage{titlesec}
\usepackage{titletoc}
\usepackage{geometry}
\usepackage{enumitem}
\usepackage{setspace}
\usepackage{amsmath}
\usepackage{listings}
\date{}
\begin{document}
\begin{doublespace}
\title{\textbf{\vspace{1em}{}}}
\maketitle

\centering
{\LARGE \bfseries Computer Science 2XC3 - Final Project}
\vspace{3em}

{\Large Due Date: April 9, 2025 WINTER}\vspace{1.5em}

{\large \textbf{Group 9}}

{\large  Wu, Peter: \texttt{wu729}~\\
  Ganesh, Vikram: \texttt{vikramgv}~\\
  \vspace{0.5em}
  Bai, Franklin: \texttt{baif4}}~\\


\end{doublespace}

\newpage

\begin{onehalfspace}

\tableofcontents
\addcontentsline{toc}{section}{\large\normalfont {Part 1, Team Charter}}
\addcontentsline{toc}{section}{\large\normalfont {Part 2, Single Source Shortest Path Algorithms }}
\addcontentsline{toc}{section}{\large\normalfont {Part 3, All-pair Shortest Path Algorithm}}
\addcontentsline{toc}{section}{\large\normalfont {Part 4, A* Algorithm} }
\addcontentsline{toc}{section}{\large\normalfont {Part 5, Compare Shortest Path Algorithms}}
\addcontentsline{toc}{section}{\large\normalfont {Part 6, Organize code per UML Diagram} }~\\
\newpage


\section*{Part 1, Team Charter}
\subsection*{Communication}
\large{For our group, we decided that communication over \textbf{Discord} would be the most convenient for all members. Discord also allows screen sharing, making it a valuable tool for collaboration. Each member is expected to respond within \textbf{one hour} to maintain efficiency and work within our given timeframe.

For planned video calls, members are expected to join within \textbf{15 minutes} of the scheduled time. Since these meetings take up a significant portion of our schedules, punctuality is appreciated. If a member is unable to join, they must notify the group beforehand so the rest can plan accordingly. Any health or family emergencies will, of course, be waivered.

\textbf{Repeated failure} to adhere to the above communication agreements will result in an email to a TA or the professor, outlining the member's inability to meet agreed timeframes. This may lead to a deduction in their grade.

\subsection*{Collaboration Tools}

Our team will use \textbf{GitHub} to maintain version control. Each member may use the IDE of their choice, provided it is properly synced with Git. A shared repository will be created to store all code, diagrams, and other essential resources. This allows us to collaborate simultaneously, monitor each other's progress, and assist remotely.

The final report will be written using \textbf{LateX} to take advantage of its formatting capabilities compared to other programs. Sections may be drafted in other software and later copied or converted into LateX.

\subsection*{Dispute Resolution}

Team disputes will be resolved via a discussion on Discord or through a scheduled video call. During this discussion, the team will review the reasons behind any disagreements. If necessary, a second chance may be offered. 

If disputes cannot be resolved within the team and repeated failures occur, the matter will be escalated to a TA or the professor via email.

\subsection*{Deadlines and Task Allocation}

Our first internal deadline is \textbf{March 27th}, during which we will have a video call to discuss our progress on individual components of the project. We will have another internal deadline around \textbf{April 7th}, 2 days before the deadline, where we will begin finalizing our project.

Initial task distribution is as follows:

\begin{itemize}[leftmargin=1.5cm]
    \item \textbf{Franklin:} Part 1, 6
    \item \textbf{Vik:} Part 2, 3
    \item \textbf{Peter:} Part 4, 5
    \item \textbf{Team:} Part 7
\end{itemize}

These assignments are not final. Members are encouraged to assist one another, as we recognize the difficulty level may vary between different parts of the project.}
\newpage

\section*{Part 2, Single Source Shortest Path Algorithms}
\newpage

\section*{Part 3, All-pair Shortest Path Algorithm}
\newpage

\section*{Part 4, A* Algorithm}
\newpage

\section*{Part 5, Compare Shortest Path Algorithms}
\newpage

\section*{Part 6, Organize code per UML Diagram}

The design principles and patterns being used in this diagram include composition, class inheritance, and adapters. 

HeuristicGraph inherits from WeightedGraph, which inherits from Graph class.

Dijkstras, Bellman\_Ford, and A\_star (which is an adapter class for original A\_star algorithm, another design principle) all inherit from SPAlgorithm.

Finally, ShortPathFinder uses composition (has-a relation) by passing a Graph object and interchangeable SPAlgorithm types into the class to help run. This way we can allow runtime switching of algorithms to whichever may be more efficient for the situation, especially using the set\_algorithm function.

Currently, nodes are represented as only integers which is very limiting in that it cannot represent much information. Thus, a solution could be implementing a custom Node class, where we can define metadata for each node, like name, value, date, or something similar. A sample implementation could be:

\begin{lstlisting}[language=Python]
class Node:
    def __init__(self, name: str, value: int, date: int):
        self.name = name
        self.value = value
        self.date = date

    def __repr__(self):
        return f"Node(name={self.name!r},
        value={self.value},date={self.date})"
\end{lstlisting}

Now whenever we need a node, we can pass in this Node object, which contains a lot more data than just a simple integer value.

This means we must change all classes that utilize nodes. We can change adjacency list from \texttt{List[int]} to \texttt{Dict[Node, List[Node]]}, or change weights to use \texttt{Dict[Tuple[Node, Node], float]}. Other sample changes include changing \texttt{add\_node(node: int)} function, to \texttt{add\_node(node: Node)}. And when we need specific values from Node, we can simply call \texttt{Node.name} or \texttt{Node.date}, etc. This would be a much better design for this UML.
%\section*{List of Figures}
%\startlist{toc}
%\printlist{toc}{section}{\textbf{Part 5, werlkwerjoik}}

\end{onehalfspace}
\end{document}
